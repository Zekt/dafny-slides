\documentclass[10pt, compress]{beamer}

\usetheme{m}

\usepackage{booktabs}
\usepackage[scale=2]{ccicons}
\usepackage{verbatimbox}

\title{Introduction to Dafny}
\subtitle{}
\date{\today}
\author{Lin Tzu-Chi}

\begin{document}

\maketitle

\begin{frame}[fragile]
  \frametitle{Dafny}

  Dafny is an imperative programming language with built-in annotations to prove correctness of code.
  \begin{itemize}
  \item Dafny is built on Boogie, a intermediate verification language.
  \item Boogie generates verification conditions, which are passed to an SMT solver (Z3 by default).
  \end{itemize}
  
\end{frame}

\section{Basic Syntax}

\begin{frame}[fragile]
  \frametitle{Methods}
  
  \verb|method|s are functions in typical imperative languages.
  
  \begin{verbnobox}[\footnotesize]
method Abs(x: int) returns (y: int)
{
   if x < 0
     { return -x; }
   else
     { return x; }
}
  \end{verbnobox}
  The input parameters are read only, and an implicit \verb|return| is added automatically at the end of a method, where the current values of return parameters are returned as-is.
\end{frame}

\begin{frame}[fragile]
  \frametitle{Methods}
  
  There can be multiple return values.
  \begin{verbnobox}[\footnotesize]
method MultipleReturns(x: int, y: int)
returns (more: int, less: int)
{
   more := x + y;
   less := x - y;
   // comments.
}
  \end{verbnobox}
\end{frame}

\begin{frame}[fragile]{Pre- and Postconditions}
  \verb|ensures| annotates postconditions of a method for Dafny to check its correctness. 
  \begin{verbnobox}[\footnotesize]
method MultipleReturns(x: int, y: int)
returns (more: int, less: int)
   ensures less < x
   ensures x < more
{
   more := x + y;
   less := x - y;
}
  \end{verbnobox}	
  
Dafny rejects this program.
\end{frame}

\begin{frame}[fragile]{Pre- and Postconditions}
  \verb|requires| annotates preconditions. It is the programmer's job to establish them.
  \begin{verbnobox}[\footnotesize]
method MultipleReturns(x: int, y: int)
returns (more: int, less: int)
   requires 0 < y
   ensures less < x < more
{
   more := x + y;
   less := x - y;
}

  \end{verbnobox}
  Dafny verifies this program successfully.

\end{frame}

\begin{frame}[fragile]{Pre- and Postconditions}
  We can check by ourselves that postcondition above holds under given precondition. But how does Dafny know that?

  Pre- and Postconditions are translated to the following formula, which is passed to an SMT solver,  ($\to$ represents implication.)
  \begin{align*}
    0 < y \to x + y > x \\
    0 < y \to x - y < x \\
  \end{align*}
  which can also be checked by hand. 
\end{frame}

\begin{frame}[fragile]{Pre- and Postconditions}
  While an SMT solver cannot directly prove that a proposition always hold, the problem can be translated to verify negations/conjunction of negations of above formulas, such that if they are unsatisfiable, there is no program state that could violates given pre- and postconditions:
    $$\neg((0 < y \to x + y > x) \land (0 < y \to x - y < x))$$
\end{frame}

\begin{frame}[fragile]
  \frametitle{Assertions}
  \verb|assert| is a keyword to place assertions in the moddle of a method.
  \begin{verbnobox}[\footnotesize]
// use definition of Abs() from before.
method Testing()
{
   var v := Abs(3);
   assert 0 <= v;
}
  \end{verbnobox}
\end{frame}

\begin{frame}[fragile]
  \frametitle{Assertions}

  The program:
  \begin{verbnobox}[\footnotesize]
var v := Abs(3);
assert v == 3;
  \end{verbnobox}
  would not be verified, because Dafny only knows the postconditions of \verb|Abs|, but nothing more.
\end{frame}

\begin{frame}[fragile]
  \frametitle{Assertions}
  To prove the assertion above, we can modify \verb|Abs| to provide more information.
  \begin{verbnobox}[\footnotesize]
method Abs(x: int) returns (y: int)
   ensures 0 <= y
   ensures 0 <= x ==> x == y
{
   // body as before
}
  \end{verbnobox}
\end{frame}

\begin{frame}[fragile]
  \frametitle{Functions}
  \begin{verbnobox}[\footnotesize]
function abs(x: int): int
{
   if x < 0 then -x else x
}
  \end{verbnobox}[\footnotesize]
  Unlike a method, which can have all sorts of statements in its body, a function body must consist of exactly one expression, with the correct return type.
\end{frame}

\begin{frame}[fragile]
  \frametitle{Functions}
The power of functions comes from the fact that they can be used directly in specifications. So we can write:
  \begin{verbnobox}[\footnotesize]
assert abs(3) == 3;
  \end{verbnobox}
without explicitly writing pre- and postconditions.

\end{frame}

\begin{frame}[fragile]{Loop Invariants}
To pre- and postconditions of a loop, We would want to konw what always hold for each loop iteration.
  \begin{verbnobox}[\footnotesize]
var i := 0;
while i < n
   invariant 0 <= i
{
   i := i + 1;
}
  \end{verbnobox}
When you specify an invariant, Dafny proves two things:
  \begin{itemize}
  \item the invariant holds upon entering the loop, and
  \item it is preserved by the loop.
  \end{itemize}
\end{frame}

\begin{frame}[fragile]{Loop Invariants}
Assuming the invariant holds at the beginning of the loop, Dafny checks that executing the loop body once makes the invariant hold again:
    $$0 \leq i \to 0 \leq i + 1$$
, as mentioned, to verify above formula, the problem is translated to unsatisfiability of the negation of the formula.

\end{frame}

\begin{frame}[fragile]{Termination}
A decreases annotation, as its name suggests, gives Dafny and expression, called measure, that strictly decreases with every loop iteration or recursive call.
  \begin{verbnobox}[\footnotesize]
while 0 < i
   invariant 0 <= i
   decreases i
{
   i := i - 1;
}
  \end{verbnobox}
Combining with the requirement that it must be an integer, the measure will reach zero in finite number of iterations, thus ensures termination.
\end{frame}

\begin{frame}[fragile]{Loop Invariants}
A decreases is automatically added if not given by the user, so an addition formula is added to verification condition whenerver a loop is present, in this case:
  $$i - 1 < i$$
\end{frame}

\begin{frame}[fragile]{Loop Invariants}
  Excercise: What is the verification condition (the formula to verify to prove correctness) of the following program?
  \begin{verbnobox}[\footnotesize]
method ComputeFib(n: nat) returns (b: nat)
   ensures b == fib(n)
{
   if n == 0 { return 0; }
   var i: int := 1;
   var a := 0;
       b := 1;
   while i < n
      invariant 0 < i <= n
      invariant a == fib(i - 1)
      invariant b == fib(i)
   {
      a, b := b, a + b;
      i := i + 1;
   }
}
  \end{verbnobox}
\end{frame}

\begin{frame}[fragile]{Look into Boogie and Z3}
  We take a look into Boogie and Z3 to understand how a program is verified in Dafny.

  Using Dafny command line tool, corresponding Boogie file of a Dafny program can be obtained,
  Boogie programs generated by Dafny are usually very complicated even for simple Dafny programs, but we can write a Boogie program that looks very similiar to Dafny for our purpose:
  \begin{verbnobox}[\footnotesize]
procedure F(n: int) returns (r: int)
  ensures 100 < n ==> r > 90;
{
  if (100 < n) {
    r := n - 10;
  } else {
    r := n + 100;
  }
}
  \end{verbnobox}
\end{frame}

\begin{frame}[fragile]{Look into Boogie and Z3}
  We can successfully verify the Boogie program above with Z3, where the verification condition passed to Z3 can be exported by the Boogie command line tool (here we only have one \verb|ensures|, thus only one condition is generated):
  \begin{verbnobox}[\tiny]
(assert (not
(let ((GeneratedUnifiedExit_correct  (=> (! (and %lbl%+0 true) :lblpos +0) (! (or %lbl%@1  (=> (< 100 n) (> r@2 90))) :lblneg @1))))
(let ((anon3_Else_correct  (=> (! (and %lbl%+2 true) :lblpos +2) (=> (<= n 100) (=> (and (= r@1 (+ n 100)) (= r@2 r@1)) GeneratedUnifiedExit_correct)))))
(let ((anon3_Then_correct  (=> (! (and %lbl%+3 true) :lblpos +3) (=> (< 100 n) (=> (and (= r@0 (- n 10)) (= r@2 r@0)) GeneratedUnifiedExit_correct)))))
(let ((anon0_correct  (=> (! (and %lbl%+4 true) :lblpos +4) (and anon3_Then_correct anon3_Else_correct))))
anon0_correct))))
))
  \end{verbnobox}

\end{frame}

\begin{frame}[fragile]{Look into Boogie and Z3}
  The above condition can be simplified as follows:
  \begin{verbnobox}[\scriptsize]
~ (l4 & T ->
    (l3 & T ->
      (100 < n  ->
        (r0 == n - 10 ->
          (l0 & T -> l1 or (100 < n -> r0 > 90))))) &
    (l2 & T ->
      (n <= 100 ->
        (r1 == n + 100 ->
          (l0 & T -> l1 or (100 < n -> r1 > 90))))))
  \end{verbnobox}
  We can check by hand to see this formula is unsatisfiable. i.e. It is impossible to have a case such that the postcondition does not hold.
\end{frame}

\begin{frame}[fragile]{Arrays}
  \verb|array<T>| is the type of array of \verb|T|. Arrays can be \verb|null|, and have a built in length field, \verb|a.Length|.
  All array accesses must be proven to be within bounds.
  This program can be successfully verified:
  \begin{verbnobox}[\footnotesize]
method Find(a: array<int>, key: int) returns (index: int)
   ensures 0 <= index ==> index < a.Length && a[index] == key
{
  return -1;
}
  \end{verbnobox}
\end{frame}

\begin{frame}[fragile]{Quantifiers}
  \begin{verbnobox}[\scriptsize]
method Find(a: array<int>, key: int) returns (index: int)
   ensures 0 <= index ==> index < a.Length && a[index] == key
   ensures index < 0 ==> forall k :: 0 <= k < a.Length ==> a[k] != key
{
   index := 0;
   while index < a.Length
      invariant 0 <= index <= a.Length
      invariant forall k :: 0 <= k < index ==> a[k] != key
   {
      if a[index] == key { return; }
      index := index + 1;
   }
   index := -1;
}
  \end{verbnobox}
\end{frame}

\begin{frame}[fragile]{Predicates}
  A predicate is a function which returns a boolean. 
  \begin{verbnobox}[\footnotesize]
predicate sorted(a: array<int>)
   requires a != null
{
   forall j, k :: 0 <= j < k < a.Length ==> a[j] <= a[k]
}
  \end{verbnobox}
Note that there is no return type, because predicates always return a boolean.
Dafny rejects this code as given, claiming that the predicate cannot read a. Fixing this issue requires another annotation, the reads annotation.
\end{frame}

\begin{frame}[fragile]{Framing}
  The sorted predicate is not able to access the array because the array was not included in the function's reading frame. The reading frame of a function (or predicate) is all the memory locations that the function is allowed to read. 
  Dafny will check that you do not read any memory location that is not stated in the reading frame.
  \begin{verbnobox}[\footnotesize]
predicate sorted(a: array<int>)
   requires a != null
   reads a
{
   forall j, k :: 0 <= j < k < a.Length ==> a[j] <= a[k]
}
  \end{verbnobox}
\end{frame}

\begin{frame}[fragile]{Binary Search}
  \begin{verbnobox}[\tiny]
predicate sorted(a: array<int>)
   requires a != null
   reads a
{
   forall j, k :: 0 <= j < k < a.Length ==> a[j] <= a[k]
}
method BinarySearch(a: array<int>, value: int) returns (index: int)
   requires a != null && 0 <= a.Length && sorted(a)
   ensures 0 <= index ==> index < a.Length && a[index] == value
   ensures index < 0 ==> forall k :: 0 <= k < a.Length ==> a[k] != value
{
   var low, high := 0, a.Length;
   while low < high
      invariant 0 <= low <= high <= a.Length
      invariant forall i ::
         0 <= i < a.Length && !(low <= i < high) ==> a[i] != value
   {
      var mid := (low + high) / 2;
      if a[mid] < value
      {
         low := mid + 1;
      }
      else if value < a[mid]
      {
         high := mid;
      }
      else
      {
         return mid;
      }
   }
   return -1;
}
  \end{verbnobox}
\end{frame}

\section{Termination}

\begin{frame}[fragile]{Termination}
There are two potential sources of non-terminating (divergent) behavior: loops, and recursive functions and methods. 
Dafny employs \verb|decreases| annotations for handling either case.

A decreases annotation specifies a value, called the termination measure, that becomes strictly smaller each time, and it is bounded.

Each kind of termination measure comes with a built-in lower bound,
Dafny proves that the termination measure gets smaller on each iteration. 
\end{frame}

\begin{frame}[fragile]
  \begin{verbnobox}[\footnotesize]
while i < n
   invariant 0 <= i <= n
{
   // do something interesting
   i := i + 1;
}
  \end{verbnobox}
Dafny verifies this program immediately. Because it is a common loop form and Dafny tries to guess the termination measure \verb|decreases n - i|.

Dafny requires that the loop does not execute again when the termination measure is negative. 
\end{frame}

\begin{frame}[fragile]
Dafny proves the termination of the whole program, not just loops.
For each function/method that is possibly recursive, it requires either an explicit or implicit decreases annotation on it.

Most recursive functions just call themselves with smaller values of the parameters, so the parameters decreasing is the default guess:
  \begin{verbnobox}[\footnotesize]
function fac(n: nat): nat
{
   if n == 0 then 1 else n * fac(n-1)
}
  \end{verbnobox}
\end{frame}

\begin{frame}[fragile]
Dafny provides a special annotation that instructs Dafny not to attempt to prove termination:
  \begin{verbnobox}[\footnotesize]
method hail(N: nat)
   decreases *
{
   var n := N;
   while 1 < n
      decreases *
   { 
      n := if n % 2 == 0 then n / 2 else n * 3 + 1;
   }
}
  \end{verbnobox}
\end{frame}

\begin{frame}[fragile]
The final termination measure is a tuple of the other kinds of measures. For example, the following implementation of the Ackermann function uses a pair of integers to prove termination:
  \begin{verbnobox}[\footnotesize]
function Ack(m: nat, n: nat): nat
   decreases m, n
{
   if m == 0 then n + 1
   else if n == 0 then Ack(m - 1, 1)
   else Ack(m - 1, Ack(m, n - 1))
}
  \end{verbnobox}
Eventually one of the elements must decrease.
\end{frame}

\begin{frame}[fragile]
Termination applies not just to single functions/methods, but also to multiple mutually recursive functions/methods:
  \begin{verbnobox}[\footnotesize]
predicate even(n: nat)
   ensures even(n) <==> n % 2 == 0
{
   if n == 0 then true else odd(n-1)
}
predicate odd(n: nat)
   ensures odd(n) <==> n % 2 != 0
{
   if n == 0 then false else even(n-1)
}
  \end{verbnobox}
\end{frame}

\section{Lemmas}

\begin{frame}[fragile]{Lemmas}
Sometimes there are steps of logic required to prove a program correct, but they are too complex for Dafny to discover and use on its own:
  \begin{verbnobox}[\scriptsize]
method FindZero(a: array<int>) returns (index: int)
   requires a != null
   requires forall i :: 0 <= i < a.Length ==> 0 <= a[i]
   requires forall i :: 0 < i < a.Length ==> a[i-1]-1 <= a[i]
   ensures index < 0  ==> forall i :: 0 <= i < a.Length ==> a[i] != 0
   ensures 0 <= index ==> index < a.Length && a[index] == 0
{
   index := 0;
   while index < a.Length
      invariant 0 <= index
      invariant forall k :: 0 <= k < index && k < a.Length ==> a[k] != 0
   {
      if a[index] == 0 { return; }
      index := index + a[index];
   }
   index := -1;
}
  \end{verbnobox}
\end{frame}

\begin{frame}[fragile]{Lemmas}
A \verb|lemma| is really just a ghost method. The desired property stated by the lemma is declared as the postcondition:
  \begin{verbnobox}[\footnotesize]
lemma Lemma(...)
   ensures (desirable property)
{
   ...
}
  \end{verbnobox}
\end{frame}

\begin{frame}[fragile]{Lemmas}
For the zero search problem:
  \begin{verbnobox}[\scriptsize]
lemma SkippingLemma(a : array<int>, j : int)
   requires a != null
   requires forall i :: 0 <= i < a.Length ==> 0 <= a[i]
   requires forall i :: 0 < i < a.Length ==> a[i-1]-1 <= a[i]
   requires 0 <= j < a.Length
   ensures forall i :: j <= i < j + a[j] && i < a.Length ==> a[i] != 0
{
   ...
}
  \end{verbnobox}
\end{frame}

\begin{frame}[fragile]{Lemmas}
Checking that this lemma is what we want:
  \begin{verbnobox}[\scriptsize]
method FindZero(a: array<int>) returns (index: int)
   requires a != null
   requires forall i :: 0 <= i < a.Length ==> 0 <= a[i]
   requires forall i :: 0 < i < a.Length ==> a[i-1]-1 <= a[i]
   ensures index < 0  ==> forall i :: 0 <= i < a.Length ==> a[i] != 0
    ensures 0 <= index ==> index < a.Length && a[index] == 0
{
   index := 0;
   while index < a.Length
      invariant 0 <= index
      invariant forall k :: 0 <= k < index && k < a.Length ==> a[k] != 0
   {
      if a[index] == 0 { return; }
      SkippingLemma(a, index);
      index := index + a[index];
   }
   index := -1;
}
  \end{verbnobox}
\end{frame}

\begin{frame}[fragile]{Lemmas}
Finishing the lemma:
  \begin{verbnobox}[\scriptsize]
lemma SkippingLemma(a : array<int>, j : int)
   requires a != null
   requires forall i :: 0 <= i < a.Length ==> 0 <= a[i]
   requires forall i :: 0 < i < a.Length ==> a[i-1]-1 <= a[i]
   requires 0 <= j < a.Length
   ensures forall i :: j <= i < j + a[j] && i < a.Length ==> a[i] != 0
{
   var i := j;
   while i < j + a[j] && i < a.Length
     invariant i < a.Length ==> a[j] - (i-j) <= a[i]
     invariant forall k :: j <= k < i && k < a.Length ==> a[k] != 0
   {
     i := i + 1;
   }
}
  \end{verbnobox}
\end{frame}

\end{document}
